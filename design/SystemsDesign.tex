\documentclass{article}

\title{Systems Design Exercise}
\author{Group 3: \\
  Saluki Hadahiro \\
  Kim-Long Vo \\
  Grace Livingston \\
  Olivia Smith \\}

\begin{document}

\maketitle
\tableofcontents

\section{Hardware Requirements}

In order to provide a platform that can manage the operation of the
drive train and the ongoing operator interfaces and still have enough
of a safety margin, a SoC with a PWM controller, 4 ADCs, 4 SPI
interfaces, an ethernet interface, and several GPIOs should be
available on the system. These are common on many platforms.

The CPU doesn’t need to be exceptionally fast. The time for each
revoloution of a 3600 RPM motor is 1.6 ms. With 12 sensors, even with
a round robin scheduler, the microcontroller would have 133 μs to
service each sensor’s data assuming each sensor needed servicing on
each revolution. But even with this rotational rate, the actual driven
system is likely going to move much slower, and if there’s a hard
response time of 100 ms from a fault detection to the application of
the emergency brake, at 8MHz that’s still 800,000 clock cycles, which
is far more than most CPUs require to service an interrupt and change
the state of a GPIO.

If we include the overhead of running the management software, and
a small RTOS on the system, a 16MHz RISC system with 512MB of RAM
would be more than enough to run everything. 4GB of flash storage for
firmware, failsafe copies, data collection, and configuration would be
enough to continue operation under most expected conditions while
still providing reliable control and management with or without
remote software.

\section{Configuration and Failsafe Operation}

The system’s configuration should be validated on load and on change
to prevent out-of-range operation from causing damage or harm.
In order to keep the configuration as a whole adequate, configuration
changes should be validated and then made to the entire configuration
at once only after validation succeeds. Invalid configurations should
maintain the last good configuration, or on startup should use
a known-safe configuration.

\section{Fault Detection}

\subsection{Controller Faults}

The system should use ECC for RAM and flash. This will ensure that
hardware data corruption is detected, and where possible, corrected.
Statistics on failures should be maintained as part of the standard
operation and management modes.

\section{Motor Control}

\subsection{PWM Controller}

The system is required to manage the motor speed and direction.
To control both, a controller with a PWM peripheral connected to an
H-Bridge will enable the system to provide positive control over both.
The signals provided to this peripheral can then be used to compute
the expected velocity, current draw, and temperature of the various
drive train components given the environmental temperature. This can
then be compared to the actual measurements to identify possible
faults and to isolate them.

\subsection{Rotational Velocity}

To measure rotational velocity, a rotational Hall Effect Sensor on the
motor shaft can be used to determine both. When the speed or direction
doesn’t match those expected, the controller can signal a motor fault
and stop operation.

These sensors normally communicate on an $I^2C$ bus. These are very
common on standard SoCs and should be standard on most controllers
considered for this application. Ideally the output of the
transmission would also have a corresponding sensor allowing the
controller to compare the input and output velocities to catch drive
train faults.

Since they also have no moving parts and no sensors that can be
occulded by dust, the maintenance requirements are exceptionally low
compared to other options. They also require less power and
calibration than other options, and device faults are much easier to
detect when they happen.

\subsection{Temperature}

The controller also is required to monitor the motor temperature.
This can be done with a series of thermocouples. Each would monitor
the temperature of a different part of the drive system. The motor
housing itself, depending on its size, should have at least two, one
for the drive spindle end and one on the opposite side of the housing.
The spindle and transmission should also each have at least one to
monitor the condition of the drive train.

For these, the thermocouples can be monitored with one or more ADCs.
To save on peripheral input requirements, multiple thermocouples can
be multiplexed together, and a set of GPIO pins can then be used to
select between them. The multiplexor will have an effect on the ADC
readings, but at the same time, the accuracy of these readings doesn’t
need to be high precision so much as the long-term trends and the
low/nominal/high temperature bands would need to be identifiable.

Having one or more measuring the ambient temperature would also
provide a stronger ongoing calibration signal for the others.
This will allow compensating for both drift over time and more
ephemeral environmental changes, and also provide more information on
the presence of sensor faults.

\subsection{Current Draw}

Another operational parameter to measure is the motor current draw.
The current draw can be monitored with another Hall Effect sensor, and
read via an ADC channel or over an $I^2C$ bus, depending on the kind
of current sensor. The higher the resolution of data available, the
easier it would be to identify test signals added to the PWM output
that wouldn’t affect the motor velocity but would still be detected by
the current probe.

On its own the current can provide much of the information the other
sensors would also provide, such as heavy load, no load, short
circuits, and so on. When combined with the velocity and thermal
information, the current draw can provide much more accurate data on
the system status, and when there is a current fault the other sensors
can then help isolate where the fault is and what kind of fault it is.

For example, if the drive train is disconnected by damage or by other
maintenance, the motor current curve will be much different, but will
still correlate similarly to the velocity curve. The thermal data will
not show as much of an increase as normal, however, and the peak
current will also be much lower, and this would then isolate the fault
to this specific category. Adding in the drive train velocity and
temperature data would then show no corresponding motion on the output
of the drive train, or would even possibly show no data from the drive
train sensors if they’re disconnected automatically with the drive
train itself, and would prevent a potentially dangerous fault where
the motor starts while exposed to people working on it in the absence
of other housing fault sensors.

\subsection{Self-Test}

All of these sensors introduce several opportunities for faults
unrelated to the actual drive train operation. However, by
coordinating the inputs, and by performing test readings from each on
start-up and while under operation, sensor faults can then also
be identified. For example, even if the motor is locked, sending
a short series of high-frequency pulses out of the PWM should be
reflected in the current monitor convolved with the inductance of the
motor’s coils, showing as an average of the pulse duty cycle.
If there’s a wiring fault in the motor, this will change or prevent
these pulses from propagating through to the current sensors.

\section{Control Interfaces}

\subsection{Local Interface}

For the local interface, we can use a capacitive touch screen with
programmable buttons. These are reliable, work in a range of
enironments, and are inexpensive because of the proliferation of
smart phones. They usually interface with UART for the programmable
buttons, and over a memory-mapped or SPI interface for the
display itself.

The programmable buttons provide a good interface for things like
emergency stop controls, immediate configuration changes, and other
user inputs that can’t be lost and that hardware buttons could lose
over time from wear and from damage.

The display also can provide context-sensitive information to the
user, like the current operational status, any warnings from the
self-test or other sensors, and especially any faults detected.
This information can be provided without needing users to navigate to
it, since the controller would know exactly what’s most urgent, or can
otherwise display the current system state. It also can provide
graphical depictions of the location of the fault to users on the
system itself to aid in troubleshooting and repair.

\subsection{Remote Interface}

Using an Ethernet interface would allow the system to integrate with
common control protocols such as SNMP, and present a control interface
using a web browser. It can also be configured to send messages via
e-mail. For more advanced networks it can even integrate with the
local authentication system.

\section{Emergency Brake}

The system should have a brake that’s engaged when there’s no power,
or when an emergency stop button is engaged. This stop button, as well
as other hardware failsafe devices, should engage the brake and notify
the controller via an interrupt that it’s been engaged. The controller
should also be able to engage this brake as part of normal operation
or in the event of an emergency condition detected by the
various sensors.

\end{document}
